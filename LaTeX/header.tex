% Einstellung der Dokumentklasse für A4 und weitere Optionen
\documentclass[a4paper, 12pt, bibliography=totoc, listof=totoc, parskip=half, numbers=noenddot, oneside]{scrreprt} % ONESIDE
%\documentclass[a4paper, 12pt, bibliography=totoc, listof=totoc, parskip=half, numbers=noenddot, twoside]{scrreprt} % TWOSIDE

% Mathematische Zeichen in der Überschrift werden ebenfalls fett gesetzt
\addtokomafont{disposition}{\boldmath} 

% scrhack ersetzt in den Packages hyperref, float und listings einige Makros für bessere KOMAScript-Kompatibilität (muss vor den betreffenden Packages aufgerufen werden)
\usepackage[]{scrhack}

% Deutsch nach neuer Rechtschreibung
\usepackage[english, ngerman]{babel}

% Automatische Ersetzung der Umlaute
\usepackage[utf8]{inputenc}

% Erweiterter Zeichensatz incl. Umlaute
\usepackage[T1]{fontenc}

%% Times als Schriftart
%\usepackage{mathptmx}
%\usepackage[scaled=.95]{helvet} 
%\usepackage{courier}

%% Latin Modern als Schriftart
%\usepackage{lmodern}

%% Arial bzw. Helvetica als Schriftart (excl. Formelsatz)
%\renewcommand{\familydefault}{\sfdefault}
%\usepackage{helvet}



% Palatino als Schriftart
\usepackage{mathpazo}
\usepackage[scaled=.95]{helvet}
\usepackage{courier}

% Kopf- und Fußzeilen
\usepackage[automark, headsepline]{scrpage2}

% Kopf- und Fußzeile definieren
\pagestyle{scrheadings}
\renewcommand{\headfont}{\bfseries}
\clearscrheadfoot
\ohead{\headmark}                              
\cfoot{\pagemark} % ONESIDE
%\ofoot{\pagemark} % TWOSIDE

% Einstellung der Seitenränder
\usepackage[includeheadfoot]{geometry}

%Seitenränder für A4
\geometry{includeheadfoot, headsep=1.7cm, left=2.75cm, right=2.25cm, top=1.4cm, bottom=1.25cm, footskip=1.3cm, footnotesep=1cm}

% Einstellen der Schriftgröße der Bildbeschriftungen
\setkomafont{captionlabel}{\itshape}
\setkomafont{caption}{\itshape}

% Einstellen der Schriftarten der Überschriften
\setkomafont{chapter}{\Large}
\setkomafont{section}{\large}
\setkomafont{subsection}{\large}

%Umbruch in Formeln
 \usepackage{eqnarray}
 
 %Zellen farblich hervorheben in Tabellen
 \usepackage{colortbl}


% Abstand zwischen Bild und Beschriftung
\setlength{\abovecaptionskip}{7pt}

% Anpassen der Trennungs-/Umbruchparameter zum Verhindern von "overfull \hbox"
\setlength\emergencystretch{3em}
\tolerance=1000

% Gliederungstiefe festlegen
\setcounter{secnumdepth}{3} 
\setcounter{tocdepth}{3} 

% Formatierung von Inhalts- und Abbildungsverzeichnis
\usepackage[titles]{tocloft}

\cftsetpnumwidth{1.0cm}

\setlength\cftchapnumwidth{0.8cm}
\setlength\cftsecnumwidth{1.2cm}
\setlength\cftsubsecnumwidth{1.5cm}
\setlength\cftsubsubsecnumwidth{1.8cm}
\setlength\cftfignumwidth{1.4cm}

\setlength\cftchapindent{0cm}
\setlength\cftsecindent{0.8cm}
\setlength\cftsubsecindent{2.0cm}
\setlength\cftsubsubsecindent{3.5cm}

\renewcommand{\cftchapaftersnum}{}
\renewcommand{\cftsecaftersnum}{}
\renewcommand{\cftsubsecaftersnum}{}
\renewcommand{\cftsubsubsecaftersnum}{}
\renewcommand{\cftfigaftersnum}{}

% Optischer Randausgleich 
\usepackage{microtype}

% Einbinden von Grafiken
\usepackage{graphicx}

% Silbentrennung in Überschriften
\usepackage{ragged2e}
\renewcommand*{\raggedsection}{\RaggedRight}     

% Grafiken mit [H] als Parameter zwingend an der Aufrufstelle platzieren
\usepackage{float}

% E-TeX aktivieren, um mehr \dimens freizugeben
%\usepackage{etex}

% Ermöglicht segmentierte Matrizen
\usepackage{easymat}

% Formelsatz in fett
\usepackage{bm}

%\usepackage{floatrow}

% Verhindert Einbinden von Grafiken vor ihrer Aufrufstelle
\usepackage{flafter}

% Mehrere Grafiken nebeneinander
\usepackage{subcaption}


% Schönerer Formelsatz
\usepackage{amsmath, amssymb}
% Komma als Dezimaltrennzeichen interpretieren
\usepackage{icomma}

% Zeichen für math. Transformationen
\usepackage{trfsigns}

% setspace für Längeneinstellungen mit \setlength usw.
\usepackage{setspace}

% Tabellensatz
\usepackage{booktabs}
\usepackage{array}
\usepackage{tabularx}
\usepackage{Tabbing}
\usepackage{longtable}
\usepackage{ltablex}
\usepackage{dcolumn} % Tabellenzeilen am Dezimalpunkt ausrichten

\newcolumntype{L}[1]{>{\raggedright\arraybackslash}p{#1}} % linksbündig mit Breitenangabe
\newcolumntype{C}[1]{>{\centering\arraybackslash}p{#1}}   % zentriert mit Breitenangabe
\newcolumntype{R}[1]{>{\raggedleft\arraybackslash}p{#1}}  % rechtsbündig mit Breitenangabe

% Listen/Aufzählungen mit Tabulatoren
\usepackage{listliketab}
\usepackage{paralist}

% make enum items configurable
\usepackage{enumitem}

% Breite horizontale Linie mit \whline (benötigt booktabs)
\newcommand\whline{\specialrule{1.5pt}{0pt}{0pt}}


% Sonderzeichen und griechische Buchstaben im normalen Text
\usepackage{textcomp}

% Euro-Zeichen €
\usepackage{eurosym}

% Zahlen mit Einheiten setzen
\usepackage{siunitx}
% \usepackage{unitsdef}

% Farbe im Dokument
\usepackage{xcolor} 

% Nummerierung von Nicht-Gleitobjekten
% \usepackage{nonfloat, capt-of}

% Formatierten Quellcode einbinden
% \usepackage{listings, keyval}

% Dirty Trick, um align-Umgebung in amsmath abkürzen zu können
% (das obsolete eqnarray funktioniert aber mit newcommand)
\def\beqar#1\eeqar{\begin{align*}#1\end{align*}}
\def\beqarn#1\eeqarn{\begin{align}#1\end{align}}

% Strafpunkte für Schusterjungen und Hurenkinder
\clubpenalty = 1000
\widowpenalty = 1000 
\displaywidowpenalty = 1000

% Grafiken werden erst ab 75% Seitenhöhe alleine auf eine float page gesetzt
\renewcommand{\floatpagefraction}{0.75}

% Floatobjekte, die alleine auf einer Seite stehen, werden nach oben statt vertikal zentriert gesetzt
\makeatletter
\setlength{\@fptop}{0pt}
\makeatother

% Theorem-Umgebung
\usepackage{amsthm}
% \newtheorem{Name der Umgebung}{Angezeigter Ausdruck}
\newtheorem{Erkenntnis}{Erkenntnis}

% Zitieren im Autor-Jahr-Schema
\usepackage[numbers]{natbib}


%Weitere Optionen für natdin_iwb.bst:
\makeatletter
\newcommand{\bibstyle@natdin}{%

%Formatierung der Literaturverweise im Fließtext
\bibpunct{(}{)}{;}{a}{}{,~}

%Literaturverzeichnis: Labels fett / Zeilenumbruch nach Labels
\gdef\NAT@biblabelnum##1{\textsc{\textbf{##1}}\\}} %

\bibstyle@natdin

% Einzug der Belege nach dem Label im Literaturverzeichnis
\setlength{\bibhang}{7mm} 

\makeatother


% Kein Seitenumbruch von Fußnoten
\interfootnotelinepenalty=10000

% Rahmen mit \begin{frame} ... \end{frame}
% \usepackage{framed}

% Benutzung von Hintergründen
%\usepackage{everyshi}
%\usepackage{eso-pic}
%\usepackage{calc}
%\usepackage{ifthen}
%\usepackage{wallpaper}

% Anführungszeichen mit \enquote
\usepackage[autostyle,german=guillemets]{csquotes}

% Einbindung von PDF
\usepackage{pdfpages}

% landscape
\usepackage{pdflscape}

% Verfasser und Betreuer ausrichten
\usepackage[absolute]{textpos}

%Tabellen- und Abbildungsnummerierung ohne Kapitelnummer
\usepackage{chngcntr}
\counterwithout{figure}{section}
\counterwithout{table}{section} 
\counterwithout{equation}{section}
\counterwithin{figure}{chapter}
\counterwithin{table}{chapter} 
\counterwithin{equation}{chapter}


% Definition eigener Befehle/Abkürzungen
\newcommand{\beq}{\begin{equation*}}
\newcommand{\eeq}{\end{equation*}}
\newcommand{\beqn}{\begin{equation}}
\newcommand{\eeqn}{\end{equation}}
\newcommand{\bssp}{\begin{spacing}{1}}
\newcommand{\essp}{\end{spacing}}
\newcommand{\nl}{\\[0,3cm]}
\newcommand{\entspricht}{\mathrel{\widehat{=}}} 
\newcommand{\diag}{\mathrm{diag}}
\newcommand{\res}{\mathrm{res}}
\newcommand{\ges}{\mathrm{ges}}
\newcommand{\reduce}{\mathrm{red}}
\newcommand{\tats}{\mathrm{tats}}
\newcommand{\soll}{\mathrm{soll}}
\newcommand{\abw}{\mathrm{abw}}
\newcommand{\ber}{\mathrm{ber}}
\newcommand{\Steuer}{\mathrm{Steuer}}
\newcommand{\Aktor}{\mathrm{Aktor}}
\newcommand{\Sensor}{\mathrm{Sensor}}
\newcommand{\Eck}{\mathrm{Eck}}
\newcommand{\Feder}{\mathrm{Feder}}
\newcommand{\krit}{\mathrm{krit}}
\newcommand{\theo}{\mathrm{theo}}
\newcommand{\real}{\mathrm{real}}
\newcommand{\Verst}{\textrm{Verstärker}}
\newcommand{\grad}{\ensuremath{^\circ}}
\newcommand{\iwb}{\textit{iwb} }


%Alle Abkürzungen müssen im Abkürzungsverzeichnis erklärt werden!
\newcommand{\zb}{z.\,B.}
\newcommand{\og}{o.\,g.}
\newcommand{\ua}{u.\,a.}
\newcommand{\ie}{d.\,h.}
\newcommand{\Dh}{D.\,h.}
\newcommand{\zt}{z.\,T.}
\newcommand{\zz}{z.\,Z.}
\newcommand{\idr}{i.\,d.\,R.}

%1,5-facher Zeilenabstand
\onehalfspacing

% Links im pdf-Dokument setzen (sollte ganz am Ende geladen werden!)
\usepackage{hyperref}

% Units verwenden
\usepackage{units}

\usepackage[miktex]{gnuplottex}

% \def\acrobat{\hyperdef{jump}{here}{}}
%Semi-automatic forward search
%An much more useful solution would be to use the hyperref package:
%
%1. Make sure the hyperref package is loaded: \usepackage{hyperref}
%
%2. Define the following command: \def\acrobat{\hyperdef{jump}{here}{}}
%
%3. Change the "view project's output" and "forward search" DDE command (as described earlier on) to: command: [DocOpen("%bm.pdf")][FileOpen("%bm.pdf")][DocGoToNameDest("%bm.pdf","jump.here")]
%
%4. Use somewhere in the text (usually, where you are currently working…) the new command \acrobat and, after (re)opening, the reader should jump to that location.
%
%Note that steps 1. and 2. have to be done in the header of your document. If you use pdfTeX, you should load the hyperref package via \usepackage[pdftex]{hyperref}; if you use dvips, load it using \usepackage[dvips]{hyperref}. If several \acrobat commands are issued, the reader always jumps to the "latest" one.

%% URLs umbrechen (nach hyperref aufrufen!)
%\usepackage{breakurl}

% closing roots http://tex.stackexchange.com/questions/29834/closed-square-root-symbol#answer-29838
\usepackage{letltxmacro}
\makeatletter
\let\oldr@@t\r@@t
\def\r@@t#1#2{%
\setbox0=\hbox{$\oldr@@t#1{#2\,}$}\dimen0=\ht0
\advance\dimen0-0.2\ht0
\setbox2=\hbox{\vrule height\ht0 depth -\dimen0}%
{\box0\lower0.4pt\box2}}
\LetLtxMacro{\oldsqrt}{\sqrt}
\renewcommand*{\sqrt}[2][\ ]{\oldsqrt[#1]{#2}}
\makeatother


\usepackage{epstopdf}

% http://ctan.org/pkg/adjustbox
\usepackage[export]{adjustbox}

% Listings
\definecolor{lst_comment}{HTML}{008000}
\definecolor{lst_string}{HTML}{A31515}
\definecolor{lst_numbers}{HTML}{2B91AF}

\usepackage{listings}
\lstset{language = C++}

 \lstset{
   basicstyle=\scriptsize\ttfamily\scriptsize,
   keywordstyle=\bfseries\ttfamily\color{blue},
   stringstyle=\color{lst_string}\ttfamily,
   commentstyle=\color{lst_comment}\ttfamily,
	 %identifierstyle=\color{blue},
   emph={square}, 
   emphstyle=\color{blue}\texttt,
   emph={[2]root,base},
   emphstyle={[2]\color{yac}\texttt},
   showstringspaces=true,
   flexiblecolumns=false,
   tabsize=4,
   numbers=left,
   numberstyle=\tiny\ttfamily\bfseries\color{lst_numbers},
   numberblanklines=false,
   stepnumber=1,
   numbersep=6pt,
   xleftmargin=17pt,
	 breaklines = true,
 }

% Weist pdflatex an, die Ausgabe in pdf-Version 1.6 vorzunehmen.
% Das vermeidet Fehler mit eingebundenen pdfs > v1.4
\pdfminorversion=6


\hypersetup{
	pdfauthor={Vorname Nachname},
	pdftitle={Titel der Arbeit},
	pdfsubject={Studienarbeit an der TU München, Lehrstuhl MiMed, Garching},
	pdfkeywords={{TUM}},
	pdfborder={0 0 0},
	pdfcreator={LaTeX}
}

%
\numberwithin{figure}{section}
\numberwithin{table}{section}
\numberwithin{equation}{section}
\AtBeginDocument{\numberwithin{lstlisting}{section}}

% take text in circles little bit lower to make it better looking

\let\textcircledold\textcircled
\let\textcircled\relax
%\newcommand{\textcircled}[1]{\raisebox{.5pt}{\textcircledold{\raisebox{-.9pt} {#1}}}}

\usepackage{tikz}
\newcommand*\textcircled[1]{\tikz[baseline=(char.base)]{
    \node[shape=circle,draw,inner sep=2pt] (char) {#1};}}

\usepackage{multicol}

% Check if all labels are referenced
%\usepackage{refcheck}

%eigene Farben
\definecolor{drot}{rgb}{0.5,0,0} 
\definecolor{dgruen}{rgb}{0,0.5,0} 
\definecolor{dblau}{rgb}{0,0,0.5}
\definecolor{lila}{rgb}{0.5,0,0.5} 
\definecolor{rosa}{rgb}{1,0,1} 
\definecolor{dgelb}{rgb}{0.5,0.5,0} 
\definecolor{turkis}{rgb}{0,1,1} 

%fuer Matlab-Quellcode-Darstellung
\usepackage[numbered,framed]{mcode}


