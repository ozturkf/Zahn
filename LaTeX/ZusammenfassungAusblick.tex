\cleardoublepage
\counterwithout{figure}{section}
\counterwithout{table}{section} 
\counterwithout{equation}{section}
\counterwithin{figure}{chapter}
\counterwithin{table}{chapter} 
\counterwithin{equation}{chapter}


\chapter{Summary and Outlook}
The coloring process of the dental crowns based on the Zirconia ceramics is a necessity because of the plain white appearance of the material with a 49\% translucency. The coloring process is realized through utilization of the metal-ionic inks. The sintering procedure binds the ceramic powder to provide a dental material with a high structural strength and also generates the color via oxidizing the metal-ionic at temperatures about 1200\textdegree C. Depending on the type of the ink classified by the groups of A, B, C and D, the color is defined and the concentration of the applied color determines the shade of the application area. Since the coloring process of the dental zirconia is not a two dimensional paint application process, but a 3 dimensional ink distribution problematic, due to the depth dependent luminescence of the dental problems, the digitalization of the procedure is not as simple as it is for a desktop printer. 

